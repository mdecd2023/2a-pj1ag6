\renewcommand{\baselinestretch}{1.5} %設定行距
\pagenumbering{roman} %設定頁數為羅馬數字
\clearpage  %設定頁數開始編譯
\sectionef
\addcontentsline{toc}{chapter}{摘~~~要} %將摘要加入目錄
\begin{center}
\LARGE\textbf{摘~~要}\\
\end{center}
\begin{flushleft}
\fontsize{14pt}{20pt}\sectionef\hspace{12pt}\quad 本學期採取個人及團體分組來學習,團體實習目標為開發一款能在 web-based CoppeliaSim 場景中雙方或多方對玩的遊戲。\\[12pt]
\fontsize{14pt}{20pt}\sectionef\hspace{12pt}\quad  除了設計 bubbleRob 機器人、繪製球場、設定感測器及寫能讓雙方對打的程式碼外,組員間也需要共同維護、分工合作。\\[12pt]	
\fontsize{14pt}{20pt}\sectionef\hspace{12pt}\quad 此專題是利用兩台 BubbleRob 雙輪車在一足球場景中進行對戰,雙方球門分別設有感測器,各有一名 BubbleRob 負責運球。 在規定時間內,每進一球即透過程式重新從球場中線發球,重啟賽局。模擬場景中還須配置計分板顯示比賽剩餘時間與比分。在 CoppeliaSim 模擬環境中進行測試運用上的可行性並嘗試透過埠號供使用者觀看。\\[12pt]

\end{flushleft}
\newpage
%=--------------------Abstract----------------------=%
\renewcommand{\baselinestretch}{1.5} %設定行距
\addcontentsline{toc}{chapter}{Abstract} %將摘要加入目錄
\begin{center}
\LARGE\textbf\sectionef{Abstract}\\
\begin{flushleft}
\fontsize{14pt}{16pt}\sectionef\hspace{12pt}\quad This semester, individual and group grouping will be adopted for learning. The group practical goal is to develop a game that can be played by two or more parties in a web-based CoppeliaSim scene.\\[12pt]
\fontsize{14pt}{16pt}\sectionef\hspace{12pt}\quad In addition to designing the bubbleRob robot, drawing the playing field, setting up sensors, and writing code that enables both parties to play, team members also need to maintain and collaborate on tasks.\\[12pt]
\fontsize{14pt}{16pt}\sectionef\hspace{12pt}\quad This project involves using two BubbleRob wheeled robots to compete in a soccer field scene, with sensors installed on each goal post. Each team has one BubbleRob responsible for ball handling. Within a specified time, each goal scored will trigger a program to reset the game from the center of the field, restarting the match. A scoreboard is also configured in the simulation scene to display the remaining time and score. The feasibility of testing and using the project in the CoppeliaSim simulation environment will be explored, and users will be able to view the simulation via port number.\\[12pt]
\end{flushleft}
